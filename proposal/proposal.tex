\documentclass{article} \usepackage[T1]{fontenc} \usepackage{amssymb, amsmath,
graphicx} \usepackage[ruled,noline]{algorithm2e} \usepackage{hyperref}
\usepackage{bbm}

\setlength{\oddsidemargin}{.25in} \setlength{\evensidemargin}{.25in}
\setlength{\textwidth}{6in} \setlength{\topmargin}{-0.4in}
\setlength{\textheight}{8.5in}

\newcommand{\heading}[6]{ \renewcommand{\thepage}{\arabic{page}} \noindent
    \begin{center} \framebox{ \vbox{ \hbox to 5.78in { \textbf{#2} \hfill #3 }
                \vspace{4mm} \hbox to 5.78in { {\Large \hfill #6  \hfill} }
                \vspace{2mm} \hbox to 5.78in { \textit{#4 \hfill #5} } } }
        \end{center} \vspace*{4mm} }

\newtheorem{theorem}{Theorem} \newtheorem{definition}[theorem]{Definition}
\newtheorem{remark}[theorem]{Remark} \newtheorem{lemma}[theorem]{Lemma}
\newtheorem{corollary}[theorem]{Corollary}
\newtheorem{proposition}[theorem]{Proposition}
\newtheorem{claim}[theorem]{Claim}
\newtheorem{observation}[theorem]{Observation} \newtheorem{fact}[theorem]{Fact}
\newtheorem{assumption}[theorem]{Assumption}

\newenvironment{proof}{\noindent{\bf Proof:} \hspace*{1mm}}{ \hspace*{\fill}
    $\Box$ } \newenvironment{proof_of}[1]{\noindent {\bf Proof of #1:}
    \hspace*{1mm}}{\hspace*{\fill} $\Box$ }
    \newenvironment{proof_claim}{\begin{quotation} \noindent}{ \hspace*{\fill}
    $\diamond$ \end{quotation}}

    \newcommand{\lecturenotes}[5]{\heading{#1}{6.824 Distributed
    Systems}{#2}{#5}{#3}{#4}}



\begin{document}
%%%%%%%%%%%%%%%%%%%%%%%%%%%%%%%%%%%%%%%%%%%%%%%%%%%%%%%%%%%%%%%%%%%%%%%%%%%%%%%
% PLEASE MODIFY THESE FIELDS AS APPROPRIATE
\newcommand{\psetnum}{3}          % lecture number
\newcommand{\lecturetitle}{Final Project Proposal} % topic of lecture
\newcommand{\lecturedate}{\today}  % date of lecture
\newcommand{\studentname}{Andrea Tacchetti and Michael Gharbi}    % full name of
 \newcommand{\collaborators}{}    % full name of student

\newcommand*\xor{\mathbin{\oplus}}
%%%%%%%%%%%%%%%%%%%%%%%%%%%%%%%%%%%%%%%%%%%%%%%%%%%%%%%%%%%%%%%%%%%%%%%%%%%%%%%


%%%%%%%%%%%%%%%%%%%%%%%%%%%%%%%%%%%%%%%%%%%%%%%%%%%%%%%%%%%%%%%%%%%%%%%%%%%%%%%
\lecturenotes{\psetnum}{\lecturedate}{\studentname}{\lecturetitle}{\collaborators}
We want to build a collaborative whiteboard where users can create simple vector
line-drawings with their friends over the internet (or a LAN at first). The high
level idea is similar to Etherpad but differs from it in two important ways.
First, it is a line-drawing application, not a text-editor, second, we don't
plan to design a centralized service but rather a peer-to-peer model.

We plan to start with simple graphic primitives, typically lines, with some
parameters such as color, curvature, starting point and end point, 
all draw and erase actions will be put in a common log and executed in order 
by each peer. Each peer will thus draw in its own graphic context based on the
logged operations.

There are a number of challenges that we foresee. We will have to have a well 
defined scheme for peers to join or leave a canvas, we will have to think about how to 
cope with failures and partitions. For example, we plan to have some common view with a list of current
participants and have each peer generate fake activity that we will use as heartbeat messages. 

We will start by focusing on consistency, if time allows we might relax our model to make the experience of 
a partitioned user not completely awful.

%Techs : how about we ask the TAs for good/easy RPC libraries for python?


%- We plan to use Cairo lib with PyCairo Python binding for the drawing part. It should be basic at first : simple lines, stroke width and color, rectangles etc.
%- RPC lib for python ?
    %+ RPyC : http://rpyc.readthedocs.org/en/latest/
    %+ RpcLib https://pypi.python.org/pypi/rpclib
    %+ JSON rpc : https://pypi.python.org/pypi/jpc
    
%Vs. 'brush' bitmap drawing ?



%Steps
%We plan to represent each line with some parameters and have each peer maintain a log of lines that have been drawn and erased. There are a number of challenges that we foresee. We will have to have a well defined scheme for a new peer to join or leave a canvas, we will have to think about how to 
%cope with failures and partitions, if time allows, we will think about how to make sure that the 
%experience of a user that ends up in a minority partition is not completely awful mofidying our consistency model.

%Handle depth foreground vs. background depending on Paxos ordering. Interesting effects? Handling transforms ? Move/Scale/Rotate ? Availability or consistency first ?

%- Network and communication layer
%- PAXOS
%- Drawing canvas
\end{document}
